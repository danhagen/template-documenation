%% Packages
\usepackage{multirow}
\usepackage{multicol}
\usepackage{siunitx}
\usepackage{sectsty}
\usepackage{graphicx}
\usepackage{caption}
\usepackage{amsfonts} 
\usepackage{amsmath}
\usepackage{amssymb}
\usepackage{hyperref}
\usepackage{empheq}  % loads »mathtools«, which in turn loads »amsmath«
\usepackage{float}
\usepackage{graphicx}
\usepackage[utf8]{inputenc}
\usepackage{natbib}
\bibliographystyle{abbrvnat}
\setcitestyle{authoryear,open={(},close={)}} %Citation-related commands
\usepackage[
    a4paper,
    % bindingoffset=0.2in,
    left=1in,right=1in,
    top=1in,bottom=1in,
    footskip=.25in
]{geometry}
\usepackage{listings}
\usepackage{fancyhdr}
\setlength{\headheight}{21.35008pt}
\usepackage{appendix}
\usepackage{framed}
\usepackage{xstring}
\usepackage{enumitem}

% This package is used for inputing code/pseudocode.
\usepackage{xcolor}
% Settings for xcolor.
\definecolor{codegreen}{rgb}{0,0.6,0}
\definecolor{codegray}{rgb}{0.5,0.5,0.5}
\definecolor{codepurple}{rgb}{0.58,0,0.82}
\definecolor{backcolour}{rgb}{0.95,0.95,0.92}
\definecolor{mylilas}{RGB}{170,55,241}

\usepackage[scaled]{beramono}
% \newcommand*{\ttdefault} %% Only if the base font of the document is to be typewriter style
\usepackage[T1]{fontenc}
\lstdefinestyle{mystyle}{language=Matlab,
    backgroundcolor=\color{backcolour}, 
    stringstyle=\color{mylilas},
    basicstyle=\small\ttfamily,
    commentstyle=\color{codegreen},
    keywordstyle=\color{blue},%
    identifierstyle=\color{black},%
    morekeywords={matlab2tikz},
    keywordstyle=[2]{\color{black}},
    morekeywords=[2]{1}, 
    emph=[1]{for,end,break},emphstyle=[1]\color{red}, %some words to emphasise
    breakatwhitespace=false,         
    breaklines=true,                 
    captionpos=b,                    
    keepspaces=true,
    showspaces=false,                
    showstringspaces=false,
    showtabs=false,                  
    tabsize=2,
    numbers=none, % left,
    numberstyle={\tiny \color{black}},% size of the numbers
    numbersep=9pt, % this defines how far the numbers are from the text
    framexleftmargin=9pt,
    xleftmargin=9pt,
    framexrightmargin=9pt,
    xrightmargin=9pt
}

\lstset{style=mystyle}

